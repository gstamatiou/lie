\documentclass{report}

% Packages for math symbols and equations
\input{packages}
\input{commands}

\usepackage{tikz-cd}


% Title page information
\title{Lie group notes}
\author{Giorgos}
\date{\today}

\begin{document}

\maketitle

\tableofcontents

\chapter*{Conventions}
\begin{itemize}
    \item By Lie group, we mean either real or complex.
    \item A closed Lie subgroup of a Lie group is a submanifold that is also a subgroup (see \cref{thm:closed_subgroup}). 
    \item A Lie subgroup is a subgroup that is also an immersed submanifold.
\end{itemize}

\chapter{Lie groups: basic definitions}
The following theorem allows us to reduce the study of Lie groups to the study of finite groups and connected Lie groups, since $G_0$ is a normal subgroup of $G$ and $G/G_0$ is a discrete group.
where $G^0$ is the identity component of $G$.
\begin{theorem}[Theorem 2.6, \cite{kirillov2008introduction} ]
    Let $G$ be a real or complex Lie group and $G^0$ its identity component.
    Then $G^0$ is a normal subgroup of $G$ and a Lie group itself, while $G/G^0$ is a discrete group.
\end{theorem}

In fact, we can reduce the case of connected Lie groups to simply connected Lie groups:
\begin{theorem}[Theorem 2.7, \cite{kirillov2008introduction}]
    Let $G$ be a connected Lie group. Then its universal cover $\tilde G$ has a canonical structure of a Lie group such that the covering map $p: \tilde G \to G$ is a homomorphism of Lie groups whose kernel is isomorphic to the fundamental group of $G$.
    Moreover, in this case, $\ker p$ is a discrete central subgroup in $\tilde G$.
\end{theorem}

We have the following connection between subgroups and Lie subgroups (i.e.\ subgroups that are also submanifolds):
\begin{theorem}[Theorem 2.8, \cite{kirillov2008introduction}]\label{thm:closed_subgroup}
    \begin{enumerate}[label=(\roman*)]
        \item Any Lie subgroup of a Lie group is closed in the topology of the ambient group.
        \item Any closed subgroup of a Lie group is a real Lie subgroup.
    \end{enumerate}
\end{theorem}

\subsection{Homogeneous spaces}
We begin by describing coset spaces of Lie groups.
\begin{theorem}[Theorem 2.11, \cite{kirillov2008introduction}]
    Let $G$ be a Lie group of dimension $n$, $H \leq G$ a closed Lie subgroup of dimension $k$.
    \begin{enumerate}[label = (\roman*)]
        \item The coset space $G/H$ has a natural structure of a manifold of dimension $n-k$ such that the canonical map $p: G \to G/H$ is a fiber bundle, with fiber diffeomorphic to $H$.
        The tangent space at the identity is isomorphic to the quotient space $T_H G/H \simeq T_eG/T_eH$.
        \item If H is a normal closed Lie subgroup then $G/H$ has a canonical Lie group structure.
    \end{enumerate}
\end{theorem}

The following is the analog of the homomorphism theorem for Lie groups:
\begin{theorem}[Theorem 2.5, \cite{kirillov2008introduction}]
    Let $f:G_1 \to G_2$ be a Lie group morphism.
    \begin{enumerate}[label = (\roman*)]
        \item $H = \ker f$ is a normal closed Lie subgroup of $G_1$ and $f$ induces an injective homomorphism $G_1/H \to G_2$ that is an immersion
        \item If moreover $\Im f$ is an embedded submanifold, then it is a closed Lie subgroup of $G_2$ and $f$ induces an isomorphism $G_1/H \to \Im f$.
    \end{enumerate}
\end{theorem}

\begin{theorem}[Theorem 2.20, \cite{kirillov2008introduction}]
    Let $G$ be a Lie group acting on a manifold $M$, and $m \in M$.
    \begin{enumerate}[label = (\roman*)]
        \item The stabilizer is a closed Lie subgroup of $G$, and the orbit map $g \mapsto g \cdot m$ induces an injective immersion $G/G_m \hookrightarrow \mathcal O_m$ whose image coincides with $\mathcal O_m$.
        \item The orbit $\mathcal O_m$ is an immersed submanifold with tangent space $T_m \mathcal O_m = T_1 G / T_1 G_m $.
        \item If the orbit is a submanifold, then the orbit map is a diffeomorphism.
    \end{enumerate}
\end{theorem}

The case of one orbit gives rise to $G$-homogeneous spaces:
\begin{theorem}[Theorem 2.22 \cite{kirillov2008introduction}]
    Let $M$ be a $G$-homogeneous space and $m \in M$.
    Then the orbit map $G \to M$ is a fiber bundle over $M$ with fiber $G_m$.
\end{theorem}

\subsection{Classical Lie groups}

\begin{definition}
    We define
    \[
    Sp(n, \mathbb K) = 
    \left\{
        A \in GL(2n, \mathbb K) \mid \omega(Ax, Ay) = \omega(x, y)
    \right\}.
    \]
    where $\omega$ is the standard symplectic form on $\mathbb K^{2n}$, which is given by
    $\omega(x, y) = x^* J y = \sum_{i=1}^n x_i y_{i+n} - x_{i+n}y_i$, 
    and 
    \[
    J = 
    \begin{pmatrix}
    0 & -I_n \\
    I_n & 0
    \end{pmatrix}.
    \]
    We also define the gorup of unitary quaternionic transformations by
    \[
    \Sp(n) = \Sp(n, \mathbb C) \cap \SU(2n).
    \]
\end{definition}

The following theorem tells us that the logarithmic map behaves well when restricted to a neighborhood of the identity in each classical group.

\begin{table}[h!]
    \centering
    \begin{tabular}{c c c c c c}
        $G$ & $O(n, \mathbb{R})$ & $SO(n, \mathbb{R})$ & $U(n)$ & $SU(n)$ & $Sp(n)$ \\
        \hline \hline
        $\mathfrak{g}$ & $x + x' = 0$ & $x + x' = 0$ & $x + x^* = 0$ & $x + x^* = 0, \ \text{tr} \, x = 0$ & $x + J^{-1}x'J = 0 \ x + x^* = 0$ \\
        $\dim G$ & $\frac{n(n-1)}{2}$ & $\frac{n(n-1)}{2}$ & $n^2$ & $n^2 - 1$ & $n(2n+1)$ \\
        $\pi_0(G)$ & $\mathbb{Z}_2$ & $\{1\}$ & $\{1\}$ & $\{1\}$ & $\{1\}$ \\
        $\pi_1(G)$ & $\mathbb{Z}_2 \ (n \ge 3)$ & $\mathbb{Z}_2 \ (n \ge 3)$ & $\mathbb{Z}$ & $\{1\}$ & $\{1\}$ \\
    \end{tabular}
    \caption{Compact classical groups}
    \label{table:classical_groups}
\end{table}

\begin{table}[h!]
    \centering
    \begin{tabular}{c c c c}
        $G$ & $GL(n, \mathbb{R})$ & $SL(n, \mathbb{R})$ & $Sp(n, \mathbb{R})$ \\
        \hline \hline
        $\mathfrak{g}$ & $\mathfrak{gl}(n, \mathbb{R})$ & $\text{tr} \, x = 0$ & $x + J^{-1}x'J = 0$ \\
        $\dim G$ & $n^2$ & $n^2 - 1$ & $n(2n + 1)$ \\
        $\pi_0(G)$ & $\mathbb{Z}_2$ & $\{1\}$ & $\{1\}$ \\
        $\pi_1(G)$ & $\mathbb{Z}_2 \ (n \ge 3)$ & $\mathbb{Z}_2 \ (n \ge 3)$ & $\mathbb{Z}$ \\
    \end{tabular}
    \caption{Noncompact real classical groups.}
    \label{table:noncompact_real_classical_groups}
\end{table}

\begin{table}[h!]
    \centering
    \begin{tabular}{c c c c c}
        $G$ & $GL(n, \mathbb{C})$ & $SL(n, \mathbb{C})$ & $O(n, \mathbb{C})$ & $SO(n, \mathbb{C})$ \\
        \hline \hline
        $\pi_0(G)$ & $\{1\}$ & $\{1\}$ & $\mathbb{Z}_2$ & $\{1\}$ \\
        $\pi_1(G)$ & $\mathbb{Z}$ & $\{1\}$ & $\mathbb{Z}_2$ & $\mathbb{Z}_2$ \\
    \end{tabular}
    \caption{Complex classical groups.}
    \label{table:complex_classical_groups}
\end{table}

\chapter{Lie groups and Lie algebras}
\section{Exponential map}
\begin{definition}[Proposition 3.1, \cite{kirillov2008introduction}]
    Let $G$ be a Lie group and $x \in \mathfrak g$.
    The one-parameter subgroup $\gamma_x: \mathbb K \to G$ is the unique Lie group morphism such that $\gamma_x'(0) = x$.
    We define the exponential map of $G$ as
    \[
    \exp(x) = \gamma_x(1)
    \]
\end{definition}
\begin{remark}
    By looking at the proof of the statements in the above definition, one can see that for $x\in \mathfrak g$, the curve
    \[
    \exp(tx) = \gamma_x(t) = \gamma_{tx}(1).
    \]
    integral curve of the left-invariant vector field $X \in \mathcal X(G)$ that satisfies
    \[
    X_e = x.
    \]
\end{remark}

The following are some properties of the exponential map:
\begin{theorem}[Theorems 3.7 and 3.36, \cite{kirillov2008introduction}]
    Let $G$ be a Lie group.
    \begin{enumerate}[label = (\roman*)]
        \item $d_{e} \exp = \id_{\mathfrak g}$
        \item The exponential map is a local diffeomorphism at $0$.
        \item For any Lie group morphism $\phi:G_1 \to G_2$, we have $d_e \phi (exp(x)) = \exp d_e (\phi(x))$ for all $x\in \mathfrak g$.
        \item For any $g \in G, x \in \mathfrak g$
        \[
        g \exp(x) g^{-1} = \exp(\Ad_g x).
        \]
        \item For $x, y \in \mathfrak g$, we have:
        \[
        \text{If } [x,y] = 0 \text{ then } e^x e^y = e^y e^x = e^{x+y}.
        \]
    \end{enumerate}
\end{theorem}

\begin{remark}
    The exponential map is not surjective in general. It is however for compact Lie groups.
\end{remark}

\begin{example}
    Let $G = SO(3, \mathbb R)$.
    Then $\mathfrak{so}(3, \mathbb R)$ consists of skew-symmetric mattrices, with basis:
    \begin{align*}
        J_1 = \begin{pmatrix} 0 & 0 & 0 \\ 0 & 0 & -1 \\ 0 & 1 & 0 \end{pmatrix}, \quad
        J_2 = \begin{pmatrix} 0 & 0 & 1 \\ 0 & 0 & 0 \\ -1 & 0 & 0 \end{pmatrix}, \quad
        J_3 = \begin{pmatrix} 0 & -1 & 0 \\ 1 & 0 & 0 \\ 0 & 0 & 0 \end{pmatrix}.
    \end{align*}
    The exponential matrix is given by
    \[
    e^{tJ_1} = 
    \begin{pmatrix}
        1 & 0 & 0 \\
        0 & \cos(t) & -\sin(t) \\
        0 & \sin(t) & \cos(t)
    \end{pmatrix},
    \]
    i.e.\ rotation around x-axis by angle t; similarly, $J_y, J_z$ generate rotations around
    $y, z$ axes.
    Elements of the form $exp(tJx), exp(tJ_y ), exp(tJ_z )$ generate
    a neighborhood of identity in $\SO(3, \mathbb R)$.
    Since $SO(3, \mathbb R)$ is connected, these elements generate the whole group. 
    For this reason, it is common to refer to $J_x , J_y , J_z$ as “infinitesimal generators” of $SO(3, \mathbb R)$. Thus,
    in a certain sense $SO(3, \mathbb R)$ is generated by three elements.
\end{example}

\appendix

\chapter{Covering theory reminder}
In this chapter we recall certain facts and definitions from basic covering theory.
A nice reference for these is the Chapter 2 from \cite{hatcher2002topology}.
\begin{definition}
    A covering map is a continuous surjective map $p: \tilde X \to X$ such that for every $x \in X$ there exists an open neighborhood $U$ of $x$ such that $p^{-1}(U)$ is a disjoint union of open sets in $\tilde X$, each of which is mapped homeomorphically onto $U$ by $p$.
\end{definition}
The prototypical example of a covering map is $p: \mathbb S^1 \to \mathbb S^1, p(z) = z^n$.

One of the particular characteristics of covering spaces are their lifting properties, that we will recall below.
\begin{proposition}[Homotopy lifting property]
    Let $p: \tilde X \to X$ be a covering space and a homotopy $f_t: Y \to X$.
    Then every lift $\tilde f_0:Y \to \tilde X$ of $f_0$ extends to a unique homotopy $\tilde f_t$ lifting $f_t$.
\end{proposition}
\begin{proof}
    See \cite[Proposition 1.30]{hatcher2002topology}.
\end{proof}
This in particular implies the path lifting property of covering spaces:
\begin{corollary}
    Let $p: \tilde X \to X$ be a covering space.
    Then for every path $\gamma: I \to X$ and every lift $\tilde x_0$ of some point $x_0 \in X$ admits a unique lift $\tilde \gamma: I \to \tilde X$ of $\gamma$ starting at $\tilde x_0$.
\end{corollary}
Both of the results above imply that path-homotopies lift to pathhomotopies, where we require for a path homotopy to keep the endpoints of paths fixed.
We also have the following corollary that is useful in proving the lifting criterion below.
\begin{corollary}
    The image subgroup $p_*(\pi_1(\tilde X,\tilde x_0))$ consists of the homotopy classes of loops in $X$ based at $x_0$ whose lifts to $\tilde X$ starting at $\tilde x_0$ are loops.
\end{corollary}
\begin{proof}
    See \cite[Corollary 1.31]{hatcher2002topology}.
\end{proof}

If we care about lifting maps and not homotopies, we have the following criterion that tells us when a lift exists.
Namely when $f$ sends loops to loops that lift to loops.
\begin{proposition}[Lifting criterion]
    Let $p: (\tilde X, \tilde x_0) \to (X,x_0)$ be a covering space and $f: (Y,y_0) \to (X,x_0)$ with $Y$ being path-connected and locally path-connected.
    Then a lift $\tilde f: (Y, y_0) \to (\tilde X, \tilde x_0)$ exists if and only if $f_*(\pi_1(Y,y_0)) \subset p_*(\pi_1(\tilde X,x_0))$.
\end{proposition}
\begin{proof}
    See \cite[Proposition 1.33]{hatcher2002topology}.
\end{proof}
And regarding uniqueness of lifts:
\begin{proposition}
    Let $p: \tilde X \to X$ be a covering space and $f: Y \to X$ be a map.
    If $Y$ is connected, then any two lifts $\tilde f_1, \tilde f_2: Y \to \tilde X$ of $f$ that coincide at one point will coincide everywhere on $Y$.  
\end{proposition}
\begin{proof}
    See \cite[Proposition 1.34]{hatcher2002topology}.
\end{proof}
\printbibliography
\end{document}