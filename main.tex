\documentclass{report}

% Packages for math symbols and equations
\input{packages}
\input{commands}

\usepackage{tikz-cd}


% Title page information
\title{Lie group notes}
\author{Giorgos}
\date{\today}

\begin{document}

\maketitle

\tableofcontents

\chapter*{Conventions}
\begin{itemize}
    \item By Lie group, we mean either real or complex.
    \item A Lie subgroup of a Lie group is called a closed Lie subgroup (see \cref{thm:closed_subgroup}). 
\end{itemize}

\chapter{Lie groups: basic definitions}
The following theorem allows us to reduce the study of Lie groups to the study of finite groups and connected Lie groups, since for a general Lie group $G$ we have
\[
    G = G^0 \times G/G^0,
\]
where $G^0$ is the identity component of $G$.
\begin{theorem}[Theorem 2.6, \cite{kirillov2008introduction} ]
    Let $G$ be a real or complex Lie group and $G^0$ its identity component.
    Then $G^0$ is a normal subgroup of $G$ and $G/G^0$ is a discrete group.
\end{theorem}

In fact, we can reduce the case of connected Lie groups to simply connected Lie groups:
\begin{theorem}[Theorem 2.7, \cite{kirillov2008introduction}]
    Let $G$ be a connected Lie group. Then its universal cover $\tilde G$ has a canonical structure of a Lie group such that the covering map $p: \tilde G \to G$ is a homomorphism of Lie groups whose kernel is isomorphic to the fundamental group of $G$.
    Moreover, in this case, $\ker p$ is a discrete central subgroup in $\tilde G$.
\end{theorem}

We have the following connection between subgroups and Lie subgroups (i.e.\ subgroups that are also submanifolds):
\begin{theorem}[Theorem 2.8, \cite{kirillov2008introduction}]\label{thm:closed_subgroup}
    \begin{itemize}
        \item Any Lie subgroup of a Lie group is closed in the topology of the ambient group.
        \item Any closed subgroup of a Lie group is a real Lie subgroup.
    \end{itemize}
\end{theorem}
\printbibliography
\end{document}