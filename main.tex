\documentclass{report}

% Packages for math symbols and equations
\usepackage[T1]{fontenc}
\usepackage[utf8]{inputenc}

\usepackage[margin=1.4in]{geometry}
\usepackage{graphicx}

\usepackage{amssymb,amsthm, amsmath}
%\usepackage{mathtools}
%\numberwithin{equation}{section}

\usepackage[
natbib,
style=alphabetic,
maxbibnames=10,  
sorting=ydnt,
url=false,
doi=false,
sortcites,
defernumbers,
backref,
backend=biber
]{biblatex}
\addbibresource{bibliography.bib}

\usepackage{hyperref}

\usepackage{todonotes}
%\setuptodonotes{inline}
\usepackage{url}

\usepackage[nameinlink, capitalise, noabbrev]{cleveref}

\usepackage{xfrac}
\usepackage{nicefrac}

\usepackage{soul}

\usepackage{bbm}

\usepackage{enumitem}

%For double brackets \llbracket \rrbracket
\usepackage{stmaryrd}

\crefname{assumption}{Assumption}{Assumptions}

\newtheorem{theorem}{Theorem}

\newtheorem{proposition}{Proposition}[section]
\newtheorem{lemma}{Lemma}[section]
\newtheorem{corollary}{Corollary}[section]
\theoremstyle{remark}
\newtheorem{remark}{Remark}[section]


\theoremstyle{definition}
\newtheorem{example}{Example}[section]
\newtheorem{counterexample}{Counterexample}[section]
\newtheorem{definition}{Definition}[section]
\newtheorem{assumption}{Assumption}
\newtheorem{question}{Question}
\newtheorem{exercise}{Exercise}
\newtheorem{fact}{Fun Fact}
%\numberwithin{equation}{section}

%\renewcommand{\baselinestretch}{2}

%\newcounter{hypcounter}

\newcommand{\N}{\mathbb{N}}
\newcommand{\Z}{\mathbb{Z}}
\newcommand{\R}{\mathbb{R}}
%\DeclareMathOperator{\dom}{dom}
\newcommand{\closure}[1]{\overline{#1}}
\newcommand{\norm}[1]{\left\Vert #1 \right\Vert}
\newcommand{\seminorm}[1]{\left[ #1 \right]}
\newcommand{\abs}[1]{\left\vert #1 \right\vert}
%\DeclareMathOperator{\divtmp}{div}
\renewcommand{\div}{\divtmp}
% \DeclareMathOperator{\argmin}{arg\,min}
% \DeclareMathOperator{\argmax}{arg\,max}
% \DeclareMathOperator{\esssup}{ess\,sup}
% \DeclareMathOperator{\essinf}{ess\,inf}
\renewcommand{\st}{\,:\,}
% \DeclareMathOperator{\supp}{supp}
\newcommand{\dx}{\,\mathrm{d}x}
\renewcommand{\d}{\,\mathrm{d}}
\newcommand{\dH}{\,\mathrm{d}\mathcal{H}^{n-1}(x)}
% \DeclareMathOperator{\sign}{sign}
\newcommand{\eps}{\varepsilon}
% \DeclareMathOperator{\dist}{dist}
% \DeclareMathOperator{\Lip}{Lip}
\newcommand{\KR}{\mathrm{KR}}
\newcommand{\C}{\mathrm{C}}
\renewcommand{\L}{\mathrm{L}}
\newcommand{\W}{\mathrm{W}}
\newcommand{\M}{\mathcal M}
\newcommand{\grad}{\nabla}
\newcommand{\hess}{\mathrm{D}^2}
\newcommand{\defeq}{:=}
% \DeclareMathOperator{\diam}{diam}
\newcommand{\Set}[1]{\left\lbrace#1\right\rbrace}
\newcommand{\scale}{\eps}
\newcommand{\res}{\delta}


\newcommand{\wto}{\rightharpoonup}
\newcommand{\wsto}{\overset{\ast}{\rightharpoonup}}
\newcommand{\strictto}{\overset{\mathrm{str}}{\rightharpoonup}}

\newcommand{\rev}{\color{magenta}}
\renewcommand{\rev}{}
\newcommand{\red}{\color{red}}
\newcommand{\blue}{\color{blue}}
\newcommand{\nc}{\normalcolor}


% Lie math operators
\DeclareMathOperator{\toledo}{T}
\DeclareMathOperator{\isom}{Isom}
\DeclareMathOperator{\bus}{b}
\DeclareMathOperator{\ii}{i}
\DeclareMathOperator{\spa}{span}
\DeclareMathOperator{\class}{C}
\DeclareMathOperator{\diam}{diam}
\DeclareMathOperator{\diag}{diag}
\DeclareMathOperator{\U}{{\mathrm{U}}}
\DeclareMathOperator{\SL}{{\mathrm{SL}}}
\DeclareMathOperator{\ssl}{{\mathfrak{sl}}}
\DeclareMathOperator{\SU}{{\mathrm{SU}}}
\DeclareMathOperator{\Sp}{{\mathrm{Sp}}}
\DeclareMathOperator{\ssp}{{\mathfrak{sp}}}
\DeclareMathOperator{\su}{{\mathfrak{su}}}
\DeclareMathOperator{\PSL}{{\mathrm{PSL}}}
\DeclareMathOperator{\GL}{{\mathrm{GL}}}
\DeclareMathOperator{\gl}{{\mathfrak{gl}}}
\DeclareMathOperator{\SO}{{\mathrm{SO}}}
\DeclareMathOperator{\sso}{{\mathfrak{so}}}
\DeclareMathOperator{\PGL}{{\mathrm{PGL}}}
\DeclareMathOperator{\PO}{{\mathrm{PO}}}
\DeclareMathOperator{\PSO}{{\mathrm{PSO}}}
\DeclareMathOperator{\im}{{\mathrm{Im}}}
\DeclareMathOperator{\id}{id}
\DeclareMathOperator{\inte}{int}
\DeclareMathOperator{\LC}{LC{}}
\DeclareMathOperator{\F}{Frenet{}}
\DeclareMathOperator{\lie}{Lie}
\DeclareMathOperator{\Ker}{Ker}
\DeclareMathOperator{\Ad}{Ad}
\DeclareMathOperator{\ad}{ad}
\DeclareMathOperator{\Hff}{dim_{Hf{}f}}
\DeclareMathOperator{\vol}{Vol}
\DeclareMathOperator{\rk}{rank}
%\DeclareMathOperator{\jac}{jac}
\DeclareMathOperator{\gap}{{\sf{gap}}}
\DeclareMathOperator{\ann}{Ann}
\DeclareMathOperator{\tr}{tr}
\DeclareMathOperator{\rad}{rad}
\DeclareMathOperator{\Sym}{Sym}

%:= sign
\newcommand{\equaldef}{\overset{\mathrm{def}}{=}}

\newcommand{\restr}{\mathbin{\vrule height 1.6ex depth 0pt width
0.13ex\vrule height 0.13ex depth 0pt width 1.3ex}}

\usepackage{tikz-cd}


% Title page information
\title{Lie group notes}
\author{Giorgos}
\date{\today}

\begin{document}

\maketitle

\tableofcontents

\chapter*{Conventions}
\begin{itemize}
    \item By Lie group, we mean either real or complex.
    \item A closed Lie subgroup of a Lie group is a submanifold that is also a subgroup (see \cref{thm:closed_subgroup}). 
    \item A Lie subgroup is a subgroup that is also an immersed submanifold.
\end{itemize}

\chapter{Lie groups: basic definitions}
We will have two notions of subgroups of Lie groups:
\begin{definition}
    Let $G$ be a complex or real Lie group and $H$ be a subgroup of $G$.
    \begin{enumerate}[label = (\roman*)]
        \item $H$ is a Lie subgroup if it is an immersed submanifold of $G$ with the multiplication and inverse maps being smooth (or analytic).
        \item $H$ is a closed Lie subgroup if it is an embedded submanifold of $G$.
    \end{enumerate}
\end{definition}
\begin{example}
    Any of the classical Lie groups are closed Lie subgroups of $\GL(n, \mathbb K)$.
    For a Lie subgroup that is not closed, consider the irrational winding on the torus, that is $G = \mathbb T^2 = \mathbb R^2/\mathbb Z^2$, and $H = f(\mathbb R)$ with $f(t,s) = (t \mod \mathbb Z, \alpha s \mod \mathbb Z)$, where $\alpha \in \mathbb R\backslash \mathbb Q$.
\end{example}
\section{Lie subgroups and quotients}
The following theorem allows us to reduce the study of Lie groups to the study of finite groups and connected Lie groups, since $G_0$ is a normal subgroup of $G$ and $G/G_0$ is a discrete group.
where $G^0$ is the identity component of $G$.
\begin{theorem}[Theorem 2.6, \cite{kirillov2008introduction}]
    Let $G$ be a real or complex Lie group and $G^0$ its identity component.
    Then $G^0$ is a normal subgroup of $G$ and a Lie group itself, while $G/G^0$ is a discrete group.
\end{theorem}

In fact, we can reduce the case of connected Lie groups to simply connected Lie groups:
\begin{theorem}[Theorem 2.7, \cite{kirillov2008introduction}]
    Let $G$ be a connected Lie group. Then its universal cover $\tilde G$ has a canonical structure of a Lie group such that the covering map $p: \tilde G \to G$ is a homomorphism of Lie groups whose kernel is isomorphic to the fundamental group of $G$.
    Moreover, in this case, $\ker p$ is a discrete central subgroup in $\tilde G$.
\end{theorem}
\begin{proof}
    The lifting property of the universal cover implies that $\tilde G$ is a Lie group.
    The kernel is discrete, being the fiber of a covering map.
    The fact that it is central follows from the more general fact that every discrete normal subgroup of a connected Lie group is central.
    To show the latter, one considers the map $G \to N, g \to g n g^{-1}$ where $N$ is the normal subgroup and $n \in N$ is some fixed element.
    Then the inverse image of any element $n' \in N$ is closed an open in $G$, so it is all of $G$ in the case where $n' = n$ and empty otherwise.
\end{proof}

\begin{example}
    In the case where $G = \mathbb T^2$ is the torus, the covering map $p: \mathbb R^2 \to \mathbb T^2$ is given by $p(t,s) = e^{2\pi i t, 2\pi i s }$, and $\ker p = \mathbb N^2 \simeq \pi_1(\mathbb T^2)$.
\end{example}

We have the following connection between subgroups and Lie subgroups (i.e.\ subgroups that are also submanifolds):
\begin{theorem}[Theorem 2.8, \cite{kirillov2008introduction}]\label{thm:closed_subgroup}
    \begin{enumerate}[label=(\roman*)]
        \item Any Lie subgroup of a Lie group is closed in the topology of the ambient group.
        \item Any closed subgroup of a Lie group is a real Lie subgroup.
    \end{enumerate}
\end{theorem}
\begin{proof}
    For the first part of the theorem, we note that $\overline H$ is a subgroup of $G$ as well.
    We claim that $H$ (and thus $Hx$ for every $x \in \overline H$ ) is open and dense in $\overline H$.
    To see this, note that $e \in \overline H$ implies that for every neighborhood $U \subseteq G$ of $e$ in $G$, $U \cap H$ is nonempty.
    In particular there exists some open set $U$ in $G$ containing $r$ such that $U \cap H \neq \emptyset$.
    Then $U\cap H$ will be a neighborhood of $e$ in $\overline H$ and $H$ is open because it can be written as the union of all subsets of the form $h(U\cap H)$ for $h \in H$.

    To conclude, note that for $x, y \in \overline H$, $Hx \cap Hy$ is dense in $\overline H$, so it is nonempty.
    This implies that $Hx = Hy = H$ and $H$ thus $H = \overline H$.
\end{proof}
\section{Homogeneous spaces}
We begin by describing coset spaces of Lie groups.
\begin{theorem}[Theorem 2.11, \cite{kirillov2008introduction}]
    Let $G$ be a Lie group of dimension $n$, $H \leq G$ a closed Lie subgroup of dimension $k$.
    \begin{enumerate}[label = (\roman*)]
        \item The coset space $G/H$ has a natural structure of a manifold of dimension $n-k$ such that the canonical map $p: G \to G/H$ is a fiber bundle, with fiber diffeomorphic to $H$.
        The tangent space at the identity is isomorphic to the quotient space $T_H G/H \simeq T_eG/T_eH$.
        \item If H is a normal closed Lie subgroup then $G/H$ has a canonical Lie group structure.
    \end{enumerate}
\end{theorem}

The following is the analog of the homomorphism theorem for Lie groups:
\begin{theorem}[Theorem 2.5, \cite{kirillov2008introduction}]
    Let $f:G_1 \to G_2$ be a Lie group morphism.
    \begin{enumerate}[label = (\roman*)]
        \item $H = \ker f$ is a normal closed Lie subgroup of $G_1$ and $f$ induces an injective homomorphism $G_1/H \to G_2$ that is an immersion
        \item If moreover $\Im f$ is an embedded submanifold, then it is a closed Lie subgroup of $G_2$ and $f$ induces an isomorphism $G_1/H \to \Im f$.
    \end{enumerate}
\end{theorem}


\begin{theorem}[Theorem 2.20, \cite{kirillov2008introduction}]
    Let $G$ be a Lie group acting on a manifold $M$, and $m \in M$.
    \begin{enumerate}[label = (\roman*)]
        \item The stabilizer $G_m$ is a closed Lie subgroup of $G$, with Lie algebra
        \[
        \mathfrak h = \{ x \in \mathfrak g \mid \rho_*(x)(m) = 0 \}.
        \]
        where $\rho: G \to \mathrm{Diff}(M)$ is the action of $G$ on $M$.
        \item The orbit map $g \mapsto g \cdot m$ induces an injective immersion $G/G_m \hookrightarrow \mathcal O_m$ whose image coincides with $\mathcal O_m$.
        \item The orbit $\mathcal O_m$ is an immersed submanifold with tangent space $T_m \mathcal O_m = \mathfrak g / \mathfrak h $.
        \item If the orbit is a submanifold, then the orbit map is a diffeomorphism.
    \end{enumerate}
\end{theorem}
\begin{proof}
See \cite[Theorem 2.20, Theorem 3.29]{kirillov2008introduction}.
\end{proof}

The case of one orbit gives rise to $G$-homogeneous spaces:
\begin{theorem}[Theorem 2.22 \cite{kirillov2008introduction}]
    Let $M$ be a $G$-homogeneous space and $m \in M$.
    Then the orbit map $G \to M$ is a fiber bundle over $M$ with fiber $G_m$.
\end{theorem}

The following proposition gives us a way to check whether a Lie group morphism is a covering map:
\begin{proposition}[Lie morphism covering map criterion]
    Let $f:G_1 \to G_2$ be a Lie group morphism such that $f_*: \mathfrak g_1 \to \mathfrak g_2$ is an isomorphism.
    Then $f$ is a covering map and $\ker f$ is a discrete central group.
\end{proposition}
\begin{proof}
    The inverse mapping theorem tells us that there exist neighborhoods $U_1, U_2$ of the identities of $G_1$ and $G_2$ respectively such that $f:U_1 \to U_2$ is a diffeomorphism.
    In particular, $f$ is surjective since its image contains $U_2$ which generates $G_2$, being a neighborhood of the identity.
    Moreover, for every $g \in G_1$ and $z_1, z_2 \in \ker f$:
    \[
        gU_1 z_1 \cap gU_2 z_2 = \{e_1\} \text{ unless } z_1 = z_2.
    \]
    Indeed, if $g u z_1 = g v z_2$ for some $u, v \in U$, then $u = v z_2 z_1^{-1}$, so $f(u) = f(v)$.
    But $f$ being injective on $U_1$, this implies $u = v$ and $z_1 = z_2$.

    Thus $f(g)U_1$ is an evenly covered neighborhood for every $f(g) \in G_2$.
    \[
    f^{-1}(f(g)U_2) = g U_1 \ker f = \bigsqcup_{z \in \ker f} g U_1 z.
    \]

    To see that $\ker f$ is discrete, note that the injectivity of $f$ implies $U_1 \cap \ker f = \{ e_1 \}$.
    Being a discrete normal subgroup, it must also be central.
\end{proof}

\section{Classical Lie groups}

\begin{definition}
    We define
    \[
    Sp(n, \mathbb K) = 
    \left\{
        A \in GL(2n, \mathbb K) \mid \omega(Ax, Ay) = \omega(x, y)
    \right\}.
    \]
    where $\omega$ is the standard symplectic form on $\mathbb K^{2n}$, which is given by
    $\omega(x, y) = x^* J y = \sum_{i=1}^n x_i y_{i+n} - x_{i+n}y_i$, 
    and 
    \[
    J = 
    \begin{pmatrix}
    0 & -I_n \\
    I_n & 0
    \end{pmatrix}.
    \]
    We also define the gorup of unitary quaternionic transformations by
    \[
    \Sp(n) = \Sp(n, \mathbb C) \cap \SU(2n).
    \]
\end{definition}

The following theorem tells us that the logarithmic map behaves well when restricted to a neighborhood of the identity in each classical group.

\begin{table}[h!]
    \centering
    \begin{tabular}{c c c c c c}
        $G$ & $O(n, \mathbb{R})$ & $SO(n, \mathbb{R})$ & $U(n)$ & $SU(n)$ & $Sp(n)$ \\
        \hline \hline
        $\mathfrak{g}$ & $x^t = -x$ & $x^t = -x$ & $x^* = -x$ & $x^* = -x, \ \text{tr}, x = 0$ & $J^{-1}x^*J = -x, x^* = -x$ \\
        $\dim G$ & $\frac{n(n-1)}{2}$ & $\frac{n(n-1)}{2}$ & $n^2$ & $n^2 - 1$ & $n(2n+1)$ \\
        $\pi_0(G)$ & $\mathbb{Z}_2$ & $\{1\}$ & $\{1\}$ & $\{1\}$ & $\{1\}$ \\
        $\pi_1(G)$ & $\mathbb{Z}_2 \ (n \ge 3)$ & $\mathbb{Z}_2 \ (n \ge 3)$ & $\mathbb{Z}$ & $\{1\}$ & $\{1\}$ \\
        $\mathfrak{g}_{\mathbb C}$ &  &  & $\mathfrak{gl}(n, \mathbb C)$ & $\mathfrak{sl}(n, \mathbb C)$ & 
    \end{tabular}
    \caption{Compact classical groups}
    \label{table:classical_groups}
\end{table}

\begin{table}[h!]
    \centering
    \begin{tabular}{c c c c}
        $G$ & $GL(n, \mathbb{R})$ & $SL(n, \mathbb{R})$ & $Sp(n, \mathbb{R})$ \\
        \hline \hline
        $\mathfrak{g}$ & $\mathfrak{gl}(n, \mathbb{R})$ & $\text{tr} \, x = 0$ & $x + J^{-1}x'J = 0$ \\
        $\dim G$ & $n^2$ & $n^2 - 1$ & $n(2n + 1)$ \\
        $\pi_0(G)$ & $\mathbb{Z}_2$ & $\{1\}$ & $\{1\}$ \\
        $\pi_1(G)$ & $\mathbb{Z}_2 \ (n \ge 3)$ & $\mathbb{Z}_2 \ (n \ge 3)$ & $\mathbb{Z}$ \\
        $\mathfrak{g}_{\mathbb C}$ &  & $\mathfrak{sl(n, \mathbb C)}$ &  
    \end{tabular}
    \caption{Noncompact real classical groups.}
    \label{table:noncompact_real_classical_groups}
\end{table}

\begin{table}[h!]
    \centering
    \begin{tabular}{c c c c c}
        $G$ & $GL(n, \mathbb{C})$ & $SL(n, \mathbb{C})$ & $O(n, \mathbb{C})$ & $SO(n, \mathbb{C})$ \\
        \hline \hline
        $\pi_0(G)$ & $\{1\}$ & $\{1\}$ & $\mathbb{Z}_2$ & $\{1\}$ \\
        $\pi_1(G)$ & $\mathbb{Z}$ & $\{1\}$ & $\mathbb{Z}_2$ & $\mathbb{Z}_2$ \\
    \end{tabular}
    \caption{Complex classical groups.}
    \label{table:complex_classical_groups}
\end{table}

\chapter{Lie groups and Lie algebras}
\section{Exponential map}
\begin{definition}[Proposition 3.1, \cite{kirillov2008introduction}]
    Let $G$ be a Lie group and $x \in \mathfrak g$.
    The one-parameter subgroup $\gamma_x: \mathbb K \to G$ is the unique Lie group morphism such that $\gamma_x'(0) = x$.
    We define the exponential map of $G$ as
    \[
    \exp(x) = \gamma_x(1)
    \]
\end{definition}
\begin{remark}
    By looking at the proof of the statements in the above definition, one can see that for $x\in \mathfrak g$, the curve
    \[
    \exp(tx) = \gamma_x(t) = \gamma_{tx}(1).
    \]
    integral curve of the left-invariant vector field $X \in \mathcal X(G)$ that satisfies
    \[
    X_e = x.
    \]
\end{remark}

The following are some properties of the exponential map:
\begin{theorem}[Theorems 3.7 and 3.36, \cite{kirillov2008introduction}]
    Let $G$ be a Lie group.
    \begin{enumerate}[label = (\roman*)]
        \item $d_{e} \exp = \id_{\mathfrak g}$
        \item The exponential map is a local diffeomorphism at $0$.
        \item For any Lie group morphism $\phi:G_1 \to G_2$, we have $d_e \phi (exp(x)) = \exp d_e (\phi(x))$ for all $x\in \mathfrak g$.
        \item For any $g \in G, x \in \mathfrak g$
        \[
        g \exp(x) g^{-1} = \exp(\Ad_g x).
        \]
        \item For $x, y \in \mathfrak g$, we have:
        \[
        \text{If } [x,y] = 0 \text{ then } e^x e^y = e^y e^x = e^{x+y}.
        \]
    \end{enumerate}
\end{theorem}

\begin{remark}
    The exponential map is not surjective in general. It is however for compact Lie groups.
\end{remark}

\begin{lemma}
    The exponential map of a connected Lie group $G$ sends generators of $\mathcal g$ to generators of $G$.
    That is, if $\{ x_1, \ldots, x_n \}$ is a basis of $\mathfrak g$, then $\left\{ \exp(t x_i) : t \in \mathbb R, i \in \llbracket 1, n \rrbracket \right\}$ is a basis of $G$.
\end{lemma}
\begin{proof}
    Consider the map $f: \mathbb R^n \to \mathbb G$ given by
    \[
    f(t_1, \ldots, t_n) = \exp(t_1 x_1) \cdots \exp(t_n x_n).
    \]
    Then $d_0 f = I_n$ in the basis with respect to the standard basis for $\mathbb R^n$ and $\{ x_1, \ldots, x_n \}$ for $\mathfrak g$.
    In particular it is surjective and by the constant rank theorem, $f(\mathbb R^n)$ contatins a neighborhood of the identity of $G$. 
\end{proof}
\begin{example}
    Let $G = SO(3, \mathbb R)$.
    Then $\mathfrak{so}(3, \mathbb R)$ consists of skew-symmetric mattrices, with basis:
    \begin{align*}
        J_1 = \begin{pmatrix} 0 & 0 & 0 \\ 0 & 0 & -1 \\ 0 & 1 & 0 \end{pmatrix}, \quad
        J_2 = \begin{pmatrix} 0 & 0 & 1 \\ 0 & 0 & 0 \\ -1 & 0 & 0 \end{pmatrix}, \quad
        J_3 = \begin{pmatrix} 0 & -1 & 0 \\ 1 & 0 & 0 \\ 0 & 0 & 0 \end{pmatrix}.
    \end{align*}
    The exponential matrix is given by
    \[
    e^{tJ_1} = 
    \begin{pmatrix}
        1 & 0 & 0 \\
        0 & \cos(t) & -\sin(t) \\
        0 & \sin(t) & \cos(t)
    \end{pmatrix},
    \]
    i.e.\ rotation around x-axis by angle t; similarly, $J_y, J_z$ generate rotations around
    $y, z$ axes.
    Elements of the form $exp(tJx), exp(tJ_y ), exp(tJ_z )$ generate
    a neighborhood of identity in $\SO(3, \mathbb R)$.
    Since $SO(3, \mathbb R)$ is connected, these elements generate the whole group. 
    For this reason, it is common to refer to $J_x , J_y , J_z$ as “infinitesimal generators” of $SO(3, \mathbb R)$. Thus,
    in a certain sense $SO(3, \mathbb R)$ is generated by three elements.
\end{example}

\begin{remark}
    To motivate the term ``infinitesimal generators'', one can think of them as directions in which one can move from the identity (using the exponential map) in order to generate the whole group.
\end{remark}

\section{The commutator}
In literature, one has at least three (and four in the case of vector fields) ways to define the commutator of some Lie group, Namely
\begin{enumerate}[label = (\roman*)]
    \item Using left-invariant vector fields (see \cite{lee2018introduction}).
    \item As the lowest order term in the logarithm of the multiplication of exponentials (see \cref{lem:commutator_definition}).
    \item As the differential of the adjoint representation.
    \item In the case of vector fields, as the differentiation of the second vector field along the first.
\end{enumerate}

For completeness, we recall that in \cite{lee2018introduction}, the commutator of a Lie group is defined as
\[
[x,y] = [X,Y]_e \text{ for } x,y \in \mathfrak g
\]
where $X,Y$ are the left-invariant vector fields for which $X_e = x, Y_e = y$, and the bracket of two vector fields is defined as $[X,Y] = XY - YX$.

In \cite{kirillov2008introduction}, the commutator is defined as the lowest order term in the logarithm of the multiplication of exponentials (see \cref{lem:commutator_definition}).
We say that a map $Q:\mathfrak g \to \mathfrak g$ is of order $k$ if $Q(t x) = t^k Q(X)$ for $t \in \mathbb R, x \in \mathfrak g$.
\begin{lemma}[Lemma 3.11, \cite{kirillov2008introduction}]\label{lem:commutator_definition}
    Let $G$ be a Lie group.
    Then there exists a neighborhood of $0 \in \mathfrak g$ and smooth (or analytic in the complex case) functions $[\cdot, \cdot], \mu: \mathfrak g \times \mathfrak g \to \mathfrak g$ such that in that neighborhood:
    \begin{align*}
        e^x e^y &= e^{\mu(x,y)}\\
        \mu(x,y) &= x + y + \frac{1}{2}[x.y] + \cdots
    \end{align*}
    with $[\cdot, \cdot]$ being a bilinear skew-symmetric form and the dots denoting terms of order higher than $3$.
\end{lemma}

The commutator has the following properties:
\begin{proposition}[Proposition 3.12, \cite{kirillov2008introduction}]
    Let $G, H$ be Lie groups and $\phi: G \to H$ a Lie group morphism.
    Then for $x,y \in \mathfrak g$ we have
    \begin{enumerate}[label = (\roman*)]
        \item $\phi_{*} [x,y] = [\phi_{*} x, \phi_{*} y]$.
        \item $\Ad_g[x,y] = [\Ad_g x, \Ad_g y]$.
        \item $e^x e^y e^{-x} e^{-y} = e^{[x,y]+\cdots}$, where dots stand for degrees higher than $2$.
        \item $\ad_x y = [x,y]$, where $\ad = d_e \Ad: \mathfrak g \to \mathfrak{gl}(\mathfrak g)$.
        \item $Ad_{e^x} = e^{\ad_x} \in \mathfrak{gl}(\mathfrak g)$
    \end{enumerate}
\end{proposition}
For instance when $G$ is commutative, the commutator is zero (and the exponential map is a homomorphism).
\begin{example}
    \begin{enumerate}[label = (\roman*)]
        \item For $\mathfrak g = \mathfrak{gl}(d, \mathbb K)$, the commutator is given by
        \[
        [x,y] = xy - yx.
        \]
        \item For a general associative algebra $A$, the commutator over $\mathcal K$, is given by the same formula.
        \item Any vector space can be made into a commutative Lie algebra by defining the commutator to be zero.
    \end{enumerate}
\end{example}

\begin{theorem}[Jacobi identity]
    Let $G$ be a real or complex Lie group and $x,y,z \in \mathfrak g$.
    Then
    \[
    [x,[y,z]] + [y,[z,x]] + [z,[x,y]] = 0.
    \]
\end{theorem}
\begin{proof}
    See \cite[Theorem 3.16]{kirillov2008introduction}.
\end{proof}
\begin{proposition}
    Differentiation at the identity induces a map
    \[
    \hom(G_1, G_2) \to \hom(\mathfrak g_1, \mathfrak g_2)
    \]
    which is injective when $G_1$ is connected.
\end{proposition}


\section{Subalgebras, ideals and center}
\begin{definition}
    A subalgebra of a Lie algebra $\mathfrak g$ is a subspace $\mathfrak h \subseteq \mathfrak g$ that is closed under the Lie bracket: $[x,y] \in \mathfrak h$ for all $x,y \in \mathfrak h$.
    An ideal is a subalgebra $\mathfrak h$ such that $[x,y] \in \mathfrak h$ for all $x \in \mathfrak h, y \in \mathfrak g$.
\end{definition}
It is easy to see that one can take quotients of Lie algebras by ideals and thus obtain Lie algebras.

The next theorem tells us that Lie subgroups correspond to subalgebras, and normal closed Lie subgroups correspond to ideals.
In the case of the latter, a converse statement gives us a way to check if a subgroup is normal by checking whether the corresponding subalgebra is an ideal.
In this way we obtain a link between an algebraic condition (being a normal subgroup) and a linear one (being an ideal).
\begin{theorem}[Theorem 3.22, \cite{kirillov2008introduction}]
    Let $G$ be a real or complex Lie group with Lie algebra $\mathfrak g$.
    Then
    \begin{enumerate}[label = (\roman*)]
        \item If $H$ is a Lie subgroup (not necessarily closed) of $G$, then $\mathfrak h$ is a subalgebra of $\mathfrak g$.
        \item If $H$ is a normal closed Lie subgroup of $G$, then $\mathfrak h$ is an ideal of $\mathfrak g$.
        \item If $H$ is a closed Lie subgroup of $G$, both $G$ and  $H$ are connected, then $\mathfrak h$ is an ideal of $\mathfrak g$ if and only if $H$ is a normal subgroup of $G$.
    \end{enumerate}
\end{theorem}


\section{Lie algebra of vector fields}
For a manifold $M$, the space $\mathrm{Vect}(M)$ can be made into a Lie algebra by directly defining the bracket of two vector fields and showing that it is skew-symmetric and bilinear.
However, one can think of $\mathrm{Diff}(M)$ as similar to a Lie group (but not quite, since it is infinite-dimensional), whose Lie algebra is $\mathrm{Vect}(M)$.
\begin{definition}
    Let $M$ be a smooth manifold and $G$ be a Lie group.
    \begin{enumerate}[label = (\roman*)]
        \item A smooth map $\rho: G \to \mathrm{Diff}(M)$ is a map that arises from a smooth action of $G$ on $M$.
        \item The Lie algebra of $\mathrm{Diff}(M)$ is the space of vector fields $\mathrm{Vect}(M)$.
        \item The exponential map of $\mathrm{Diff}(M)$ is the flow (whenever well-defined) of a vector field: $\left(e^X\right)_m = \Phi^1_X(m)$.
        \item The differential of $\rho: G \to \mathrm{Diff}(M)$ is the map $\rho_*: \mathfrak g \to \mathrm{Vect}(M)$ given by
        \[
        \left(\rho_*(x)\right)_m = \left.\frac{\d}{\d t}\right|_{t=0} \rho\left(e^{tx}\right)(m).
        \]
        \item The commutator $[\xi, \eta]$ of two vector fields $\xi, \eta \in \mathrm{Vect}(M)$ is the unique vector field that satisfies
        \[
        \Phi^t_\xi \Phi^s_\eta \Phi^{-t}_\xi \Phi^{-s}_\eta = \Phi^{ts}_{[\xi, \eta]} + \cdots
        \]
        where the dots stand for terms of order 3 and higher in $s,t$.
    \end{enumerate}
\end{definition}

\begin{remark}
    To motivate why $\mathrm{Vect}(M) = \mathfrak{diff}(M)$, we note that $\mathfrak{diff}(M)$ should be given by the derivatives of the one-parameter subgroups of $\mathrm{Diff}(M)$.
    If $\phi^t \in \mathrm{Diff}(M)$ is a one-parameter subgroup of $M$, then its derivative defines a vector field $\partial_{t=0} \phi^t \in \mathrm{Vect}(M)$:
    \[
    \frac{\d}{\d t}\Big|_{t=0} \phi^t(m) \in T_m(M), \text{ for } m \in M.
    \]
    Similarly, to motivate the differential of $\rho: G \to \mathrm{Diff}(M)$, we note that $\rho$ maps one-parameter subgroups to one-parameter subgroups.
\end{remark}

\begin{proposition}[Proposition 3.23, \cite{kirillov2008introduction}]
    Let $M$ be a smooth manifold and $G$ be a Lie group acting smoothly on $M$.
    \begin{enumerate}[label = (\roman*)]
        \item The commutator of vector fields makes $\mathrm{Vect}(M)$ into a Lie algebra.
        \item The commutator can be also defined by the following formulas:
        \begin{align*}
            [\xi, \eta] &= \frac{\d}{\d t} (\Phi^t_\xi)_* \eta \\
            \partial_{[\xi, \eta]}f &= \partial_\eta \partial_\xi f - \partial_\xi \partial_\eta f\\
            \left[ f^i \partial_i, g^j \partial_j \right] &= \left( g^i \partial_i f^j - f^i \partial_i g^j \right) \partial_j
        \end{align*}
    \end{enumerate}
\end{proposition}
Note that the first expression from the second point of the above theorem should be read as
\[
[\xi, \eta]_p = \frac{\d}{\d t}\Big|_{t=0} d_{\Phi^{-t}_\xi(p)} \Phi^t_\xi (\eta_{\Phi^{-t}_\xi(p)})
\]
which allows us to interpret the Lie bracket of two vector fields as differentiation of the one along (the integral curves of) the other.

\begin{theorem}[Theorem 3.25, \cite{kirillov2008introduction}]
    The transformation $\rho_*: \mathfrak g \to \mathrm{Vect}(M)$ is a Lie algebra morphism.
\end{theorem}

\begin{example}
    \begin{enumerate}[label = (\roman*)]
        \item Considering the action of $G$ on itself by left-multiplication $L: G \to \mathrm{Diff}(G)$, the Lie algebra isomorphism
        \[
        L_*: \mathfrak g \leftrightarrow \{\text{right invariant vector fields on } \} G
        \]
        associates to $x \in \mathfrak g$ the right-invariant vector field $X$ such that $X_e = x$.
    \end{enumerate}
\end{example}

\section{Stabilizers and the center}
\begin{corollary}
    Let $f: G_1 \to G_2$ be a morphism of real or complex Lie groups.
    Then 
    \begin{enumerate}[label = (\roman*)]
        \item $\ker f $ is a closed Lie subgroup with Lie algebra $\ker f_*$.
        \item The map $G_1/\ker f \to \mathrm{Im} f$ is an immersion.
        \item When $\mathrm{Im}f$ is a submanifold, the map is a $G_1/\ker f \to \mathrm{Im}f$ diffeomorphism.
    \end{enumerate}
\end{corollary}

\begin{example}
    \begin{enumerate}[label = (\roman*)]
        \item 
        Let $V$ be a vector space over $\mathbb K$ with a bilinear form $B$.
        Then
        \[
        \mathrm{O}(V, B) = \left\{ g \in \GL(V) : B(gu, gv) = B(u,v) \text{ for all } u, v \in V \right\}
        \]
        is a closed Lie group with Lie algebra:
        \[
        \mathfrak{o}(V,B) = \left\{ g \in \mathfrak{gl}(V): B(xu, xv) = B(u,v) \text{ for all } u,v \in V\right\}.
        \]
    
        Indeed, define an action of $\GL(V)$ on the space $\mathcal F$ of bilinear forms on $V$ by $gF(u,v) = F(g^{-1}u, g^{-1}v)$.
        This defines a representation $\rho: \GL(V) \to \GL(\mathcal F)$ and $\mathrm{O}(V,B)$ is the stabilizer of $B$ under this action.
        The space $\mathcal F$ is a vector space, so $\mathrm{Vect}(\mathcal F)$ can be identifies with the space smooth maps from $\mathcal F$ to itself.
        Then for $x \in \mathfrak{gl}(V)$
        \begin{align*}
            \left(\rho_*(x)\right)_B(u,v) &= \left( \frac{\d}{\d t}\Big|_{t=0} \rho(e^{tx}) B\right)(u, v) = \left( \frac{\d}{\d t}\Big|_{t=0} B(e^{-tx}u, e^{-tx}v)\right)\\
            &= -B(u, xv) - B(xu, v).
        \end{align*}
    
        \item 
        Let $A$ be a finite-dimensional associative algebra over $\mathbb{K}$. Then the group of all automorphisms of $A$
        \[
        \text{Aut}(A) = \{g \in \text{GL}(A) \mid (g a) \cdot (g b) = g(a \cdot b) \text{ for all } a,b \in A\}
        \]
        is a Lie group with Lie algebra
        \[
        \text{Der}(A) = \{x \in \mathfrak{gl}(A) \mid (x.a)b + a(x.b) = x(ab) \text{ for all } a, b \in A\}
        \]
        (this Lie algebra is called the \textit{algebra of derivations of} $A$).
        
        Indeed, if we consider the space $W$ of all linear maps $A \otimes A \to A$ and define the action of $G$ by $(g f)(a \otimes b) = g f(g^{-1}a \otimes g^{-1}b)$ then $\text{Aut} A = G_\mu$, where $\mu : A \otimes A \to A$ is the multiplication. So $\text{Aut}(A)$ is a Lie group with Lie algebra $\text{Der}(A)$.
        \item 
        The same argument also shows that for a finite-dimensional Lie algebra $\mathfrak{g}$, the group
        \[
        \text{Aut}(\mathfrak{g}) = \{ g \in \text{GL}(\mathfrak{g}) \mid [ga, gb] = g[a, b] \text{ for all } a, b \in \mathfrak{g} \}
        \]
        is a Lie group with Lie algebra
        \[
        \text{Der}(\mathfrak{g}) = \{ x \in \mathfrak{gl}(\mathfrak{g}) \mid [x.a, b] + [a, x.b] = x.[a, b] \text{ for all } a, b \in \mathfrak{g} \}
        \]
        called the Lie algebra of derivations of $\mathfrak{g}$.
    \end{enumerate}
\end{example}


Finally, we can show that the center of $G$ is a closed Lie subgroup.


\begin{theorem}
Let $G$ be a connected Lie group. Then its center $Z(G)$ is a closed Lie subgroup with Lie algebra the ideal 
\[
\mathfrak{z}(\mathfrak{g}) = \{ x \in \mathfrak{g} \mid [x, y] = 0 \ \forall y \in \mathfrak{g} \}.
\]
\end{theorem}

The quotient group $G / Z(G)$ is usually called the \textit{adjoint group} associated with $G$ and denoted $\text{Ad}\, G$:
\[
\text{Ad}\, G = G / Z(G) = \text{Im}(\text{Ad}: G \rightarrow \text{GL}(\mathfrak{g})) \quad \text{(for connected } G).
\]
The corresponding Lie algebra is
\[
\text{ad}\, \mathfrak{g} = \mathfrak{g} / \mathfrak{z}(\mathfrak{g}) = \text{Im}(\text{ad}: \mathfrak{g} \rightarrow \mathfrak{gl}(\mathfrak{g})).
\]


\section{Campbell--Hausdorff formula}
So far, we have shown that the multiplication in a Lie group $G$ defines the
commutator in $g = T_1 G$. 
However, the definition of commutator only used the lowest non-trivial term of the group law in logarithmic coordinates. Thus, it might be expected that higher terms give more operations on g. However, it turns out that it is not so: the whole group law is completely determined by the lowest term, i.e.\ by the commutator, as follows from the Campbell--Hausdorff formula:
\begin{theorem}[Campbell--Hausdorff Formula]
    For small enough $x, y \in \mathfrak{g}$ one has
    \[
    \exp(x) \exp(y) = \exp(\mu(x, y))
    \]
    for some $\mathfrak{g}$-valued function $\mu(x, y)$ which is given by the following series convergent in some neighborhood of $(0, 0)$:
    \[
    \mu(x, y) = x + y + \sum_{n \geq 2} \mu_n(x, y), 
    \]
    where $\mu_n(x, y)$ is a Lie polynomial in $x, y$ of degree $n$, i.e. an expression consisting of commutators of $x, y$, their commutators, etc., of total degree $n$ in $x, y$. This expression is universal: it does not depend on the Lie algebra $\mathfrak{g}$ or on the choice of $x, y$.
    In particular, the group operation in a connected Lie group $G$ can be
    recovered from the commutator in $\mathfrak g$.
    \end{theorem}
    
    It is possible to write the expression for $\mu$ explicitly. 
    However, this is rarely useful, so we will only write the first few terms:
    \[
    \mu(x, y) = x + y + \frac{1}{2}[x, y] + \frac{1}{12} \left( [x, [x, y]] + [y, [y, x]] \right) + \cdots 
    \]
    

Although it follows from the Campbell--Hausdorff formula, the commutation criterion of the exponential can be also proven directly:
\begin{theorem}[Theorem 3.36, \cite{kirillov2008introduction}]
    Let $x, y \in \mathfrak g$.
    Then $e^x e^y = e^y e^x$ if and only if $[x,y]=0$.
    In that case $e^x e^y = e^y e^x = e^{x+y}$.
\end{theorem}
\begin{proof}
    The if direction follows from the definition of the commutator via Taylor series.
    For the other direction we follow the method suggested in Exercise 3.11 of \cite{kirillov2008introduction}:
    \[
    \Ad(e^x)y = e^{\ad x} y = \sum_{n\geq 0} \frac{(\ad x)^n}{n!} y = y
    \]
    But $C_g e^z = e^{\Ad_g z}$, so the above implies that $e^x e^y e^{-x} = C_{e^x}e^y = e^{\Ad(e^x)y} = e^y$.
\end{proof}

\section{Fundamental theorems of Lie theory}
Let us summarize the results we have so far about the relation between Lie
groups and Lie algebras.

\begin{enumerate}
    \item Every real or complex Lie group $G$ defines a Lie algebra $\mathfrak{g} = T_1 G$ (respectively, real or complex), with commutator defined by (3.2); we will write $\mathfrak{g} = \text{Lie}(G)$. Every morphism of Lie groups $\varphi : G_1 \to G_2$ defines a morphism of Lie algebras $\varphi_* : \mathfrak{g}_1 \to \mathfrak{g}_2$. For connected $G_1$, the map
    \[
    \text{Hom}(G_1, G_2) \to \text{Hom}(\mathfrak{g}_1, \mathfrak{g}_2)
    \]
    \[
    \varphi \mapsto \varphi_*
    \]
    is injective. (Here $\text{Hom}(\mathfrak{g}_1, \mathfrak{g}_2)$ is the set of Lie algebra morphisms.)
    \item As a special case of the previous, every Lie subgroup $H \subset G$ defines a Lie subalgebra $\mathfrak{h} \subset \mathfrak{g}$.
    \item The group law in a connected Lie group $G$ can be recovered from the commutator in $\mathfrak{g}$; however, we do not yet know whether we can also recover the topology of $G$ from $\mathfrak{g}$.
\end{enumerate}

However, this still leaves a number of questions.
\begin{question}
    Given a morphism of Lie algebras $\mathfrak{g}_1 \to \mathfrak{g}_2$, where $\mathfrak{g}_1 = \text{Lie}(G_1)$, $\mathfrak{g}_2 = \text{Lie}(G_2)$, can this morphism always be lifted to a morphism of the Lie groups?
\end{question}
\begin{question}
    Given a Lie subalgebra $\mathfrak{h} \subset \mathfrak{g} = \text{Lie}(G)$, does there always exist a corresponding Lie subgroup $H \subset G$?
\end{question}
\begin{question}
    Can every Lie algebra be obtained as a Lie algebra of a Lie group?
\end{question}

As the following example shows, in this form the answer to question 1 is negative.

\begin{example}
Let $G_1 = S^1 = \mathbb{R}/\mathbb{Z}, G_2 = \mathbb{R}$. Then the Lie algebras are $\mathfrak{g}_1 = \mathfrak{g}_2 = \mathbb{R}$ with zero commutator. Consider the identity map $\mathfrak{g}_1 \to \mathfrak{g}_2 : a \mapsto a$. Then the lift $\tilde f: \mathbb R \to \mathbb R$ of the corresponding morphism to the universal covering group (provided that the underlying morphism $f: \mathbb S^1 \to \mathbb R$ exists), should be given by $\tilde(\theta) = \theta + c$; on the other hand, it must also satisfy $\tilde f(\mathbb{Z}) = \{0\}$. Thus, this morphism of Lie algebras can not be lifted to a morphism of Lie groups.
\end{example}

In this example the difficulty arose because $G_1$ was not simply-connected. It turns out that this is the only difficulty: after taking care of this, the answers to all the questions posed above are positive. The following theorems give precise statements.

\begin{theorem}[First fundemental theorem of Lie theory]
For any real or complex Lie group $G$, there is a bijection between connected Lie subgroups $H \subset G$ and Lie subalgebras $\mathfrak{h} \subset \mathfrak{g}$, given by $H \mapsto \mathfrak{h} = \text{Lie}(H) = T_1 H$.
\end{theorem}

\begin{theorem}[Second fundemental theorem of Lie theory]
If $G_1, G_2$ are Lie groups (real or complex) and $G_1$ is connected and simply connected, then $\text{Hom}(G_1, G_2) = \text{Hom}(\mathfrak{g}_1, \mathfrak{g}_2)$, where $\mathfrak{g}_1, \mathfrak{g}_2$ are Lie algebras of $G_1, G_2$ respectively.
\end{theorem}

\begin{theorem}[Lie's third theorem]
Any finite-dimensional real or complex Lie algebra is isomorphic to a Lie algebra of a Lie group (respectively, real or complex).
\end{theorem}

These are the fundamental theorems of Lie theory and combining these with the previous results, we get the following important corollary.

\begin{corollary}[Le group and algebra equivalence]
    The categories of finite-dimensional Lie algebras and connected, simply-connected Lie groups are equivalent.
    In other words, for any real or complex finite-dimensional Lie algebra $\mathfrak{g}$, there is a unique (up to isomorphism) connected simply-connected Lie group $G$ (respectively, real or complex) with $\text{Lie}(G) = \mathfrak{g}$. Any other connected Lie group $G'$ with Lie algebra $\mathfrak{g}$ must be of the form $G/Z$ for some discrete central subgroup $Z \subset G$.
\end{corollary}

\section{Complex and real forms}
The operation of complexification, which is trivial at the level of Lie algebras,
is highly non-trivial at the level of Lie groups.
Real forms and their complexifications can be topologically quite different: for example, $\SU(n)$ is compact while $\SL(n, C)$ is not. On the other hand, it is natural to expect that their algebras share many algebraic properties, such as semisimplicity.
Thus we may use the properties of the complex Lie algebra to study the real Lie algebra, and vice versa, or use the properties of one real form to study another real form.
Thus, we may use, for example, use the compact group $\SU(n)$ to prove some results
about the non-compact group $\SL(n, C)$. Moreover, since $\mathfrak{sl}(n, \mathbb R)C = \mathfrak{sl}(n, \mathbb C)$,
this will also give us results about the non-compact real group $\SL(n, \mathbb R)$. 

\begin{definition}
    \begin{enumerate}[label = (\roman*)]
        \item 
        The complexification $\mathfrak g_{\mathbb C}$ of a real Lie algebra $\mathfrak g$ is the complex Lie algebra
    \[
    \mathfrak g_{\mathbb C} \equaldef \mathfrak g \otimes_{\mathbb R} \mathbb C = \mathfrak g \oplus i \mathfrak g.
    \]
    We say that $\mathfrak g$ is a real form of $\mathfrak g_{\mathbb C}$.
    \item
    Let $G$ be a connected complex Lie group with Lie algebra $\mathfrak g$. 
    A closed real Lie subgroup $K \leq G$ is a real form of $G$ if its Lie algebra $\mathfrak k$ is a real form of $\mathfrak g$.
    \end{enumerate}
\end{definition}

In some cases, complexification is obvious, as in $\mathfrak g = sl(n, R)$.
Others however, are less obvious, as $\mathfrak g = \mathfrak u(n)$.
Finding a real form for a complex group is can be often direct:
\begin{proposition}
    Let $G$ be a connected simply-connected complex Lie group with Lie algebra $\mathfrak g$.
    Then every real form $\mathfrak k$ of $\mathfrak g$ is realised as the Lie algebra of a real form $K$ of $G$.
\end{proposition}
\begin{proof}
    Considering $\mathfrak g = \mathfrak k \oplus i \mathfrak k$, we define the real Lie algebra automorphism
    \[
    \theta: \mathfrak g \to \mathfrak g, \quad x + iy \mapsto x - iy.
    \]
    Since $G$ is connected and simply connected, the second fundamental theorem of Lie groups tells us that $\theta$ can be lifted to a Lie group automorphism $\Theta: G \to G$.
    Then $K = G^\Theta$ is a real form of $G$ with Lie algebra $\ker \theta = \mathfrak k$.
\end{proof}
Going in the opposite direction, from a real Lie group to a complex one, is
more subtle: there are real Lie groups that can not be obtained as real forms
of a complex Lie group (for example, it is known that the universal cover
of $\SL(2, \mathbb R)$ is not a real form of any complex Lie group). It is still possible
to define a complexification for any real Lie group G; however, in general the former does not admit the latter as a subgroup.


\chapter{Representations of Lie groups and Lie algebras}
In this chapter, we will discuss the representation theory of Lie groups and Lie
algebras. 
Unless specified otherwise, all Lie groups, algebras,
and representations are finite-dimensional, and all representations are complex.
Lie groups and Lie algebras can be either real or complex; unless specified
otherwise, all results are valid both for the real and complex case.

\section{Basic definitions}

\begin{definition}
    A representation of a Lie group \( G \) is a vector space \( V \) together with a morphism \( \rho : G \to \mathrm{GL}(V) \).
    
    A representation of a Lie algebra \( \mathfrak{g} \) is a vector space \( V \) together with a morphism \( \rho : \mathfrak{g} \to \mathfrak{gl}(V) \).
    
    A morphism between two representations \( V, W \) of the same group \( G \) is a linear map \( f : V \to W \) which commutes with the action of \( G \): \( f \rho(g) = \rho(g)f \). In a similar way, one defines a morphism of representations of a Lie algebra. The space of all \( G \)-morphisms (respectively, \( \mathfrak{g} \)-morphisms) between \( V \) and \( W \) will be denoted by \( \mathrm{Hom}_G(V, W) \) (respectively, \( \mathrm{Hom}_{\mathfrak{g}}(V, W) \)).
\end{definition}
    
\begin{remark}
Morphisms between representations are also frequently called \textit{intertwining operators} because they ``intertwine'' action of \( G \) in \( V \) and \( W \).
\end{remark}

The notion of a representation is completely parallel to the notion of module over an associative ring or algebra; the difference of terminology is due to historical reasons. In fact, it is also usual to use the word ``module'' rather than ``representation'' for Lie algebras: a module over Lie algebra \( \mathfrak{g} \) is the same as a representation of \( \mathfrak{g} \). We will use both terms interchangeably.

Note that in this definition we did not specify whether \( V \) and \( G \) are real or complex. Usually if \( G \) (respectively, \( \mathfrak{g} \)) is complex, then \( V \) should also be taken a complex vector space. However, it also makes sense to take complex \( V \) even if \( G \) is real: in this case we require that the morphism \( G \to \mathrm{GL}(V) \) be smooth, considering \( \mathrm{GL}(V) \) as \( 2n^2 \)-dimensional real manifold. Similarly, for real Lie algebras we can consider complex representations requiring that \( \rho : \mathfrak{g} \to \mathfrak{gl}(V) \) be \( \mathbb{R} \)-linear.
    
Of course, we could also restrict ourselves to consideration of real representations of real groups. However, it turns out that the introduction of complex representations significantly simplifies the theory even for real groups and algebras. Thus, from now on, all representations will be complex unless specified otherwise.
    
\begin{example}
    For a Lie group $G$ or a Lie algebra $\mathfrak g$ we always have the following two representations:
    \begin{enumerate}[label = (\roman*)]
        \item Trivial representation: $V = \mathbb C, \rho(g) = \id$ for all $g \in G$ (respectively $\rho(x) = 0$ for all $x \in \mathfrak g$). 
        \item Adjoint representation: $V = \mathfrak g, \rho(g) = \Ad_g$ for all $g \in G$ (respectively $\rho(x) = \ad_x$ for all $x \in \mathfrak g$). 
        \item Coadjoint representation: $V = \mathfrak g^*, \rho(x) = \ad_x^* $ for all $x \in \mathfrak g$.
    \end{enumerate}
\end{example}

The first important result about representations of Lie groups and Lie algebras is the following theorem.
It tells us that from a representation (or a morphism of representations) of a Lie group we can obtain a representation (or a morphism of representations) of its Lie algebra, and that when $G$ is connected, simply connected, then representing $G$ and representing its Lie algebra is equivalent.
\begin{theorem}
Let \( G \) be a Lie group (real or complex) with Lie algebra \( \mathfrak{g} \).
\begin{enumerate}
    \item Every representation \( \rho : G \to \mathrm{GL}(V) \) defines a representation \( \rho_* : \mathfrak{g} \to \mathfrak{gl}(V) \), and every morphism of representations of \( G \) is automatically a morphism of representations of \( \mathfrak{g} \).
    \item If \( G \) is connected, simply-connected, then \( \rho \mapsto \rho_* \) gives an equivalence of categories of representations of \( G \) and representations of \( \mathfrak{g} \). In particular, every representation of \( \mathfrak{g} \) can be uniquely lifted to a representation of \( G \), and \( \mathrm{Hom}_G(V, W) = \mathrm{Hom}_{\mathfrak{g}}(V, W) \).
\end{enumerate}
\end{theorem}
This is an important result, as Lie algebras are, after all, finite dimensional vector spaces, so they are easier to deal with.
\begin{example}
    A representation of \( \mathrm{SU}(2) \) is the same as a representation of \( \mathfrak{su}(2) \), i.e. a vector space with three endomorphisms \( X, Y, Z \), satisfying commutation relations \( XY - YX = Z, YZ - ZY = X, ZX - XZ = Y \).
\end{example}
\begin{remark}
    This theorem can also be used to describe representations of a group which is connected but not simply-connected, in the sense that representations of $G$ are the same as representations of its universal cover that are trivial over the fundamentla group of $G$.
    More concretely, any such group $G$ can be written as \( G = \tilde{G}/Z \) for some simply-connected group \( \tilde{G} \) and a discrete central subgroup \( Z \subset G \). Thus, representations of \( G \) are the same as representations of \( \tilde{G} \) satisfying \( \rho(Z) = \mathrm{id} \). An important example of this is when \( G = \mathrm{SO}(3, \mathbb{R}) \), \( \tilde{G} = \mathrm{SU}(2) \).
\end{remark}

\begin{lemma}
    Let $\mathfrak g $ be a real Lie algebra and $\mathfrak g_{\mathbb C}$ its complexification.
    Then the categories of complex representations of $mathfrak g$ and $\mathfrak g_{\mathbb C}$ are equivalent.
\end{lemma}
The next example shows that we can reduce the problem of study of representations of a non-compact Lie group $\SL(2, \mathbb C)$ to 
\begin{example}
    The categories of finite-dimensional representations of $\SL(2, \mathbb C)$,
$\SU(2)$, $\mathfrak{sl}(2, \mathbb C)$ and $\su(2)$ are all equivalent. Indeed,
$\mathfrak{sl}(2, C) = (\su(2))_{\mathbb C}$ , so categories of their finite-dimensional representations are equivalent; since the groups $\SU(2), \SL(2, C)$ are simply-connected, they
have the same representations as the corresponding Lie algebras.
\end{example}

\section{Operations on representations}
\subsection{Subrepresentations and quotients}

\begin{definition}
    Let $V$ be a representation of $G$ (respectively $\mathfrak g$). 
    A subrepresentation is a vector subspace $W \subseteq V$ stable under the action:
    $\rho(g)W \subseteq W$ for all $g \in G$ (respectively, $\rho(x)W \subseteq W$ for all $x \in \mathfrak g$).
\end{definition}

When $W \subseteq V$ is a subrepresentation, then one can obtain the factor representation or quotient representation $V/W$ given by
\[
\rho_{V/W}(g) = \rho_V(g) + W.
\]


\subsection{Direct sum and tensor product}
\begin{lemma}\label{lem:direct_sum_tensor_product}
    Let $V , W$ be representations of $G$ (respectively, $\mathfrak g$). 
    Then there is a canonical structure of a representation on $V^*, V \oplus W, V \otimes W$.
\end{lemma}
\begin{proof}
    It is given by
    \begin{align*}
        \rho(g)(v + w) &= \rho(g)(v) + \rho(g)(w) \text{ on } V \oplus W \text{ for } G,\\
        \rho(x)(v + w) &= \rho(x)(v) + \rho(x)(w) \text{ on } V \oplus W \text{ for } \mathfrak g,\\
        \rho(g)(v \otimes w) &= \rho(g)(v) \otimes \rho(g)(w) \text{ on } V \otimes W \text{ for } G,\\
        \rho(x)(v \otimes w) &= \rho(x)(v) \otimes w + v \otimes \rho(x)(w) \text{ on } V \otimes W \text{ for } \mathfrak g,\\
        \rho_{V^*}(g) &= \rho(g^{-1})^t \text{ on } V^* \text{ for } G,\\
        \rho_{V^*}(x) &= -\rho(x)^t \text{ on } V^* \text{ for } \mathfrak g,
    \end{align*}
    where we denote with $A^t$ the adjoint opearor $A^t:V^* \to V^*$ for $A: V \to V$.
\end{proof}
For example, the coadjoint representation is obtained by dualizing the adjoint representation of $\mathfrak g$.

Before giving the next series of examples, recall that for any finite dimensional vector spaces $V, W$, we have the following cannonical isomorphism:
\begin{align*}
    V^* \otimes W &\simeq \hom(V, W)\\
    f \otimes w &\mapsto \left( u \mapsto f(u)w \right)\\
    e^i \otimes (Ae_i) &\mapsfrom A
\end{align*}
where $e_1, \ldots, e_n$ are any basis of $V$ and $e_1, \cdots, e^n$ are its dual basis for $V^*$.

\begin{example}
    \begin{enumerate}[label = (\roman*)]
        \item Coadjoint representation on $\mathfrak g^*$: Dualizing the adjoint representation of $\mathfrak g$ we obtain the coadjoint representation given by $\rho(x) = ad_x^*$
        \item $\hom(V,W)$: Given two representations $V, W$ of $G$, the space of linear maps $\hom(V,W)$ is a representation by:
        \begin{align*}
            \rho(g)_{\hom(V,W)}(A) &= \rho_V(g)A\rho_W(g^{-1}) \text{ for } g \in G,\\
            \rho(x)_{\hom(V,W)}(A) &= \rho_V(x)A - A\rho_W(x) \text{ for } x \in \mathfrak g.
        \end{align*}
        This is obtained by dualizing and tensoring as in the lemma above, i.e.\ the diagram below is commutative\\
        \begin{center}
            \begin{tikzcd}
                V \otimes W^* \arrow[r, "\simeq"] \arrow[d, "\rho(g)"] & \hom(V,W) \arrow[d, "\rho(g)"] \\
                V \otimes W^* \arrow[r, "\simeq"] & \hom(V,W)
            \end{tikzcd}    
        \end{center}
        \item Bilinear forms: Given a representation $V$, the space of bilinear forms on $V$ becomes a representation when we consider it as a subspace of $V^* \otimes V^*$. It is given by:
        \begin{align*}
            g B(u, v) &= B(g^{-1}u, g^{-1}v), \text{ for } g \in G\\
            x B(u, v) &= -B(x u, v) - B(u, x v), \text{ for } x \in \mathfrak g.    
        \end{align*}
    \end{enumerate}
\end{example}

\subsection{Invariants}
\begin{definition}
    Let $V$ be a representation of a Lie group $G$. A vector $v \in V$ is
    called invariant if $\rho(g)v = v$ for all $g \in G$. The subspace of invariant vectors
    in $V$ is denoted by $V^G$ .
    Similarly, let $V$ be a representation of a Lie algebra $\mathfrak g$. A vector $v \in V$ is
    called invariant if $\rho(x)v = 0$ for all $x \in g$. The subspace of invariant vectors
    in $V$ is denoted by $V^{\mathfrak g}$ .
\end{definition}

\begin{remark}
    When $G$ is connected, and $V$ is a representation of $G$, then
    \[
    V^G = V^{\mathfrak g}
    \]
\end{remark}

\begin{example}
    \begin{enumerate}[label = (\roman*)]
        \item $\hom(V, W)$: Considering $\hom(V,W)$ as the representation arising from two representations $V, W$, we have $\hom(V, W)^G = \hom_G(V, W)$.
        \item Bilinear forms: We have $B \in V^G$ if and only if the linear map $V \to V^*$, given by $v \to B(v, -)$, is a representations morphism.
    \end{enumerate}
\end{example}

\appendix

\chapter{Covering theory reminder}
In this chapter we recall certain facts and definitions from basic covering theory.
A nice reference for these is the Chapter 2 from \cite{hatcher2002topology}.
We begin by defining covering spaces.
\begin{definition}
    A covering map is a continuous surjective map $p: \tilde X \to X$ such that for every $x \in X$ there exists an open neighborhood $U$ of $x$ such that $p^{-1}(U)$ is a disjoint union of open sets in $\tilde X$, each of which is mapped homeomorphically onto $U$ by $p$.
\end{definition}
The prototypical example of a covering map is $p: \mathbb S^1 \to \mathbb S^1, p(z) = z^n$.

\section{Lifting properties}
One of the particular characteristics of covering spaces are their lifting properties, that we will recall below.
\begin{proposition}[Homotopy lifting property]
    Let $p: \tilde X \to X$ be a covering space and a homotopy $f_t: Y \to X$.
    Then every lift $\tilde f_0:Y \to \tilde X$ of $f_0$ extends to a unique homotopy $\tilde f_t$ lifting $f_t$.
\end{proposition}
\begin{proof}
    See \cite[Proposition 1.30]{hatcher2002topology}.
\end{proof}
This in particular implies the path lifting property of covering spaces:
\begin{corollary}
    Let $p: \tilde X \to X$ be a covering space.
    Then for every path $\gamma: I \to X$ and every lift $\tilde x_0$ of some point $x_0 \in X$ admits a unique lift $\tilde \gamma: I \to \tilde X$ of $\gamma$ starting at $\tilde x_0$.
\end{corollary}
Both of the results above imply that path-homotopies lift to pathhomotopies, where we require for a path homotopy to keep the endpoints of paths fixed.
We also have the following corollary that is useful in proving the lifting criterion below.
\begin{corollary}
    The image subgroup $p_*(\pi_1(\tilde X,\tilde x_0))$ consists of the homotopy classes of loops in $X$ based at $x_0$ whose lifts to $\tilde X$ starting at $\tilde x_0$ are loops.
\end{corollary}
\begin{proof}
    See \cite[Corollary 1.31]{hatcher2002topology}.
\end{proof}
If we care about lifting maps and not homotopies, we have the following criterion that tells us when a lift exists.
Namely when $f$ sends loops to loops that lift to loops.
\begin{proposition}[Lifting criterion]
    Let $p: (\tilde X, \tilde x_0) \to (X,x_0)$ be a covering space and $f: (Y,y_0) \to (X,x_0)$ with $Y$ being path-connected and locally path-connected.
    Then a lift $\tilde f: (Y, y_0) \to (\tilde X, \tilde x_0)$ exists if and only if $f_*(\pi_1(Y,y_0)) \subset p_*(\pi_1(\tilde X,x_0))$.
\end{proposition}
\begin{proof}
    See \cite[Proposition 1.33]{hatcher2002topology}.
\end{proof}
And regarding uniqueness of lifts:
\begin{proposition}
    Let $p: \tilde X \to X$ be a covering space and $f: Y \to X$ be a map.
    If $Y$ is connected, then any two lifts $\tilde f_1, \tilde f_2: Y \to \tilde X$ of $f$ that coincide at one point will coincide everywhere on $Y$.  
\end{proposition}
\begin{proof}
    See \cite[Proposition 1.34]{hatcher2002topology}.
\end{proof}

\section{Universal covering}
In this section we will be concerned with proving that under mild conditions, a space has a universal covering, i.e.\ a simply connected covering space.
\begin{assumption}
We consider a topological space $X$ that will be path-connected, locally connected and semi-locally simply connected.
\end{assumption}
While the first two assummptions may seem rather natural, we will now explain the third one.
\begin{definition}
    A space $X$ is semi-locally simply connected if for every $x \in X$ there exists an open neighborhood $U$ of $x$ such that every loop in $U$ based at $x$ is homotopic in $X$ to a constant loop in $U$.
    In other words, the homomorphism
    \[
    \pi_1(U, x) \to \pi_1(X, x)
    \]
    induced by the inclusion is trivial.
\end{definition}
To motivate this assumption, we note that it is necessary for the existence of a universal covering.
Indeed, if $p: \tilde X \to X$ is a universal covering, $x_0 \in X$ and $U$ an evenly covered neighborhood of $x_0$, then any loop in $U$ based at $x$ lifts to a loop in $\tilde X$ based at some $\tilde x_0$.
Since $\tilde X$ is simply connected, this loop is homotopic to a constant loop in $\tilde X$ and the homotopy projects down to a homotopy in $X$.

Before constructing the universal cover, we remark that every universal cover $\tilde X$ can be thought of as homotopy classes of paths in $X$ starting at some fixed point $x_0$.
\begin{remark}
    Let $p: \tilde X \to X$ be a universal covering and $x_0 \in X$.
    Then the set of homotopy classes of paths in $X$ starting at $x_0$ is in bijection with $\tilde X$.
    Indeed, given a path $\gamma: I \to X$ starting at $x_0$, we associate it to its endpoint, which defines a map $\tilde \gamma: I \to \tilde X$.
    This map is well-defined since homotopic paths have the same endpoint.
    It is surjective, beccause $X$ is path-connected, while injectivity follows from the fact that $\tilde X$ is simply connected.
\end{remark}

We now proceed with the construction of the universal cover by fixing some $x_0 \in X$ and letting
\[
\tilde X \equaldef \left\{ [\gamma] \mid \gamma \text{ is a path in } X \text{ starting at } x_0 \right\},
\]
and defining the projection map $p: \tilde X \to X$ by $p\left([\gamma]\right) = \gamma(1)$.

To define a topology on $\tilde X$, we first consider a convenient basis of open sets for $X$:
\[
\mathcal U \equaldef \left\{ U \subseteq X \text{ open and path-connected} \mid \pi_1(U, x) \to \pi_1(X,x) \text{ is trivial for some } x \in U \right\}.
\]
Note that the above set is well-defined since if there exists some $x\in U$ such that $\pi_1(U,x) \to \pi_1(X,x)$ is trivial, then it is trivial for all $x' \in U$ because $U$ is path-connected.
To see that it is a basis for $X$,  we note that $V \in \mathcal U$ for every a path-connected open subset $V\subseteq U$.

To each homotopy class $[\gamma] \in \tilde X$ and $U \in \mathcal U$ we associate the set
\[
U_{[\gamma]} \equaldef \left\{ [\gamma \cdot \eta] \in \tilde X \mid \eta \text{ is a path in } U \text{ such that } \eta(0) = \gamma(1) \right\}.
\]
Then, given $x \in X$ and a neighborhood $U$ of $x$, the collection
\[
U_{[\gamma]} \text{ for paths } \gamma \text{ going from } x_0 \text{ to } x  
\]
will be the sheets of $U$.
The following property is crucial into showing that the topology defined by the above basis is well-defined:
\[
U[\gamma] = U_[\gamma'] \text{ for all } [\gamma'] \in U_{[\gamma]}.
\]
The proofs that $p:U_[\gamma] \to U$ is a homeomorphism for all $U \in \mathcal U$ and $[\gamma] \in \tilde X$ and that $\tilde X$ is simply connected can be found in \cite{hatcher2002topology}.

An elementary example is the torus:
\begin{example}
    For $X = \mathbb T^2$ being the torus, the universal covering $p: \mathbb R^2 \to \mathbb T^2$ and given by $p(t,s) = e^{2\pi i t, 2\pi i s }$.
    Under the identification $\pi_1(\mathbb T^2) \simeq \mathbb N^2$, which acts on $\mathbb R^2$ by translations, we have $\mathbb T^2 \simeq \mathbb R^2 / \mathbb N^2$.
\end{example}

Having now constructed the universal cover, we now move on to discussing deck transformations.
\begin{definition}
    A deck transformation of a covering space $p: \tilde X \to X$ is a homeomorphism $\phi: \tilde X \to \tilde X$ such that $p \circ \phi = p$.
    In other words, the following triangular diagram commutes:
    \[
    \begin{tikzcd}
        \tilde X \arrow[rr, "\phi"] \arrow[rd, "p"'] & & \tilde X \arrow[ld, "p"] \\
        & X &
    \end{tikzcd}
    \]
\end{definition}
For instance, the fundamental group $\pi_1(X, x_0)$ of $X$ acts by deck transformations on the universal cover $\tilde X$.
To each $[\alpha] \in \pi_1(X, x_0)$ we associate the deck transformation $\phi_{[\alpha]}$ defined by
\[
\phi_{[\alpha]}([\eta]) = [\alpha \cdot \eta].
\]
Considering the quotient space $\tilde X / \pi_1(X, x_0)$, we have the following result:
\begin{theorem}
    The quotient space $\tilde X / \pi_1(X, x_0)$ is homeomorphic to $X$ where the homeomorphism is induced by the projection map $p: \tilde X \to X$ as in the following diagram
    \[
    \begin{tikzcd}
        \tilde X \arrow[r, "p"] \arrow[d] & X \\
        \tilde X / \pi_1(X, x_0) \arrow[ur, "\simeq"', dashed]
    \end{tikzcd}
    \]
\end{theorem}

\chapter{Exercises}
In this chapter, we have solutions for certain exercises of \cite{kirillov2008introduction}.

\begin{exercise}[Exercise 2.7]
Define a bilinear form on $\mathfrak{su}(2)$ by $(a, b) = \tr(ab^*)/2$. 
Show that this form is symmetric, positive definite, and invariant under the adjoint action of
$\SU(2)$.
\end{exercise}
\begin{proof}[Solution]
    It is symmetric since in $\su(2)$ we have $b^* = -b$, so $(a,b) = \tr(ab^*)/2 = -\tr(ab)/2 = 
    -\tr(ba)/2 = (b,a)$.
    It is positive definite since $(a,a) = \frac{1}{2} \sum_{i,j} |a_{ij}|^2$.
    It is also invariant under the adjoint action of $\SU(2)$ since the trace is invariant under conjugation, and the adjoint representation is given by conjugation.
\end{proof}

\begin{exercise}[Exercise 2.8]
    Consider the usual basis for $\mathfrak{su}(2)$:
    \[
    i \sigma_1 = \begin{pmatrix} 0 & i \\ i & 0 \end{pmatrix}, \quad 
    i\sigma_2 = \begin{pmatrix} 0 & -1 \\ 1 & 0 \end{pmatrix}, \quad 
    i \sigma_3 = \begin{pmatrix} i & 0 \\ 0 & -i \end{pmatrix}.
    \]
    Show that the adjoint representation
    \begin{align*}
        \Ad: \SU(2) &\to \GL(3, \mathbb R)\\
        g &\mapsto \text{ matrix of } \Ad_g \text{ in the basis } i\sigma_1, i\sigma_2, i\sigma_3
    \end{align*}
    gives a morphism of Lie groups $\SU(2) \to \SO(3)$.
\end{exercise}
\begin{proof}[Solution]
    Note that the above exercise gave us an isometry between $\mathfrak{su}(2)$ and $\mathbb R^3$, where the first is equipped with the inner product $(a,b) = \tr(ab^*)/2$ and the second with the standard euclidean inner product.
    Denoting with $[\Ad_g]$ the matrix of $\Ad_g$ in the basis $i \sigma_1, i \sigma_2, i \sigma_3$, the above isometry tells us $\Ad_g$ preserves the given inner product if and only if the matrix $[\Ad_g]$ preserves the standard euclidean inner product, i.e.\ $[Ad_g] \in \SO(3, \mathbb R)$.
    Indeed, denoting the isometry with $\Phi: \SU(2) \to \mathbb R^2$, it gives us an isomorphism 
    
    \begin{align*}
        \GL(\Phi): \GL(\SU(2)) &\overset{\simeq}{\to} \GL(3, \mathbb R)\\
        T &\mapsto \Phi T \Phi^{-1}
    \end{align*}
    and $T \in \U(2)$ if and only if $\GL(\Phi) \in \mathrm{O}(3)$.
    But we saw in the above exercise that the inner product is preserved, so the adjoint representation lies in $\mathrm{O}(3)$.

    To show that it lies actually in $\SO(3)$, note that the determinant of every matrix in $\mathrm{O}(3)$ is either $1$ or $-1$.
    However, $\SU(2)$ is connected, so the same is true for $\phi(\SU(2))$.
    Since it is connected and contains $I_3$, so the determinant of the adjoint representation must be $1$.
\end{proof}

\begin{exercise}[Exercise 2.9]
    Let $\phi: \SU(2) \to \SO(3, \mathbb R)$ be the morphism of Lie groups defined in the previous exercise.
    Compute explicitly the map map of the tangent spaces $\phi_*: \mathfrak{su}(2) \to \mathfrak{so}(3, \mathbb R)$ and show that it is an isomorphism.
    Deduce from this that $\ker \phi$ is a discrete normal subgroup in $\SU(2)$ and that $\Im \phi$ is an open subgroup in $\SO(3, \mathbb R)$.
\end{exercise}
\begin{proof}[Solution]
    Let $Phi: \SU(2) \to \mathbb R^3$ be the isometry that maps $i\sigma_1, i \sigma_2, i\sigma_3$ to the standard basis of $\mathbb R^3$ and the induced isomohpsism $\GL(\Phi): \GL(\SU(2)) \to \GL(3, \mathbb R), \GL(\Phi)(T) = \Phi T \Phi^{-1}$.
    Then $Phi$ is given as the composition:
    \[\begin{array}{ccccc}
        \SU(2) &\to \quad \GL(\SU(2)) &\overset{\GL(\Phi)}{\to} &\GL(3, \mathbb R) \\
        g &\mapsto \quad \Ad_g &\mapsto &\Phi \Ad_g \Phi^{-1}
    \end{array}\]

    Diferentiating the above, we see that $\phi_* = \mathfrak{gl}(\Phi) \circ \ad$ which is simply the matrix of $\ad$ in the base $i\sigma_1, i\sigma_2, i\sigma_3$.
    Hence we calculate
    \[
    ad_{i\sigma_1} = - 2J_x, \quad
    ad_{i\sigma_2} = - 2J_y, \quad
    ad_{i\sigma_3} = - 2J_z.
    \]
    and thus the matrix of $\phi_* = -2 \diag(1,1,1)$ which is clearly an isomorphism.
    Hence $\phi$ is a covering map and its kernel is a discrete central normal subgroup of $\SU(2)$, and its image is an open subgroup of $\SO(3, \mathbb R)$.
\end{proof}

\begin{exercise}
    Prove that the map $\phi$ used in the two previous exercises establishes an isomorphism $\SU(2)/\mathbb Z^2 \to \SO(3, \mathbb R)$, and thus, since $\SU(2)\simeq \mathbb S^3$, we have $\mathbb{RP}^3 \simeq \SO(3, \mathbb R)$.
\end{exercise}
\begin{exercise}[Exercise 3.12]
        Let $x, y \in \mathfrak g$.
        Then $e^x e^y = e^y e^x$ if and only if $[x,y]=0$.
        In that case $e^x e^y = e^y e^x = e^{x+y}$.    
\end{exercise}
\begin{proof}[Solution]
    The if direction follows from the definition of the commutator via Taylor series.
    For the other direction we follow the method suggested in Exercise 3.11 of \cite{kirillov2008introduction}:
    \[
    \Ad(e^x)y = e^{\ad x} y = \sum_{n\geq 0} \frac{(\ad x)^n}{n!} y = y
    \]
    But $C_g e^z = e^{\Ad_g z}$, so the above implies that $e^x e^y e^{-x} = C_{e^x}e^y = e^{\Ad(e^x)y} = e^y$.
\end{proof}

\begin{exercise}[Exercise 3.15]
    Let $G$ be a connected simply-connected complex Lie group with Lie algebra $\mathfrak g$.
    Then every real form $\mathfrak k$ of $\mathfrak g$ is realised as the Lie algebra of a real form $K$ of $G$.
\end{exercise}
\begin{proof}[Solution]
    Considering $\mathfrak g = \mathfrak k \oplus i \mathfrak k$, we define the real Lie algebra automorphism
    \[
    \theta: \mathfrak g \to \mathfrak g, \quad x + iy \mapsto x - iy.
    \]
    Since $G$ is connected and simply connected, the second fundamental theorem of Lie groups tells us that $\theta$ can be lifted to a Lie group automorphism $\Theta: G \to G$.
    Then $K = G^\Theta$ is a real form of $G$ with Lie algebra $\ker \theta = \mathfrak k$.
\end{proof}

\chapter{Geomtric realisation}

\begin{proposition}
    $\SU(2)$ is homeomorphic to $\mathbb S^4$.
    In particular, it is simply connected.
\end{proposition}

\printbibliography
\end{document}